\documentclass{beamer} %

%%%BASICS
\usepackage[utf8]{inputenc}
\usepackage{csquotes}
\usepackage{enumerate}
\usepackage{mathtools}
\usepackage{qtree}
\usepackage{tikz}
\usetikzlibrary{trees}
\usepackage{amsfonts}
\usepackage{hyperref}

\usepackage{multicol}
\usepackage{graphicx}
\usepackage{epigraph}

\renewcommand{\epigraphflush}{center}
\renewcommand{\epigraphwidth}{300}


%%%Commandes

\newcommand{\N}{\mathbb{N}}
\newcommand{\Q}{\mathbb{Q}}

%%%START THEME SETTINGS
\usetheme{Berlin}
\usecolortheme{whale}
\usefonttheme{professionalfonts}
\setbeamertemplate{itemize item}{\color{black} $\blacksquare$}
%%%END THEME SETTINGS

%%%START APA
\usepackage[british]{babel}
\usepackage[backend=biber,style=apa]{biblatex}
\DeclareLanguageMapping{british}{british-apa}
\addbibresource{references.bib}
%% APA citing
%% \cite{t} - Uthor und Richter, 2010
%% \textcite{t} - Uthor und Riter (2010)
%% \parencite{t} - (Uthor & Riter, 2010)
%% \parencite[Chapt.~4]{t} - (Uthor & Riter, 2010, S. 15)
%%%END APA


\title[]{Determinacy in and out second order arithmetic}
\subtitle[]{An introduction to the proof theoretic strength of the determinacy scale}
\institute[Proof Theory Conference]{}
\author{Thibaut Kouptchinsky}

\date{December 22, 2022}

\begin{document}

\begin{frame}
	\titlepage
\end{frame}

%------------------------------------------------------

\begin{frame}{Programme}
    \tableofcontents
\end{frame}
% Presentation structure

%--------------------------------------------------------

\section{Who am I?}

\begin{frame}
    
\end{frame}


%
\section{Overview}

\begin{frame}

\end{frame}

%----------------------------------------------------------------

\section{Inside second order arithmetic}

\begin{frame}{}
    
\end{frame}

%----------------------------------------------------------

\section{Outside}

\begin{frame}{}

\end{frame}

%----------------------------------------------------------------

\end{document}
