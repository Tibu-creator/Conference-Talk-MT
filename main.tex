\documentclass{beamer} %

%%%BASICS
\usepackage[utf8]{inputenc}
\usepackage{csquotes}
\usepackage{enumerate}
\usepackage{mathtools}
\usepackage{qtree}
\usepackage{tikz}
\usetikzlibrary{trees}
\usepackage{amsfonts}
\usepackage{hyperref}

\usepackage{multicol}
\usepackage{graphicx}
\usepackage{epigraph}

\renewcommand{\epigraphflush}{center}
\renewcommand{\epigraphwidth}{300}


%%%Commandes

\newcommand{\N}{\mathbb{N}}
\newcommand{\Q}{\mathbb{Q}}

%%%START THEME SETTINGS
\usetheme{Berlin}
\usecolortheme{whale}
\usefonttheme{professionalfonts}
\setbeamertemplate{itemize item}{\color{black} $\blacksquare$}
%%%END THEME SETTINGS

%%%START APA
\usepackage[british]{babel}
\usepackage[backend=biber,style=apa]{biblatex}
\DeclareLanguageMapping{british}{british-apa}
\addbibresource{references.bib}
%% APA citing
%% \cite{t} - Uthor und Richter, 2010
%% \textcite{t} - Uthor und Riter (2010)
%% \parencite{t} - (Uthor & Riter, 2010)
%% \parencite[Chapt.~4]{t} - (Uthor & Riter, 2010, S. 15)
%%%END APA


\title[Axiomes et arithmétisation pour Cantor-Bendixson]{Mathématiques inversées : Axiomes, arithmétisation et théorème de Cantor-Bendixson}
\institute[LMAT2165]{Projet personnel et séminaire de master 1}
\author{Thibaut Kouptchinsky}

\date{Mars 2022}

\begin{document}

\begin{frame}
	\titlepage
\end{frame}

%------------------------------------------------------

\begin{frame}{Programme}
    \tableofcontents
\end{frame}
% Presentation structure

%--------------------------------------------------------

\section{Un mot d'introduction}



\begin{frame}{\textbf{The main question}: Quels sont les axiomes appropriés au développement des mathématiques ? }

        \epigraph{\textit{Fondations of mathematics is the study of the most basic concepts and logical structure of mathematics, with an eye to the unity of human knowledge}}{\textit{Stephen G. Simpson \\ SSOA (second edition)}}
\pause Aristote, Euclide, Descartes, Cauchy, Weierstraß, Dedekind, Peano, Frege, Russell, Cantor, Hilbert, Brouwer, Weyl, von Neumann, Skolem, Tarski, Heyting, Gödel, \dots 

\end{frame}


%
\section{\'Enoncé du théorème}
\begin{frame}
    \frametitle{Quelques définitions classiques}
    \begin{block}{Espace polonais}<1->
        Soit $X$, un ensemble, $X$ est dit \textbf{polonais} s'il est \begin{enumerate}
            \item Un espace métrique complet ;
            \item Séparable.
        \end{enumerate}
    \end{block}
    \begin{block}{Ensemble dénombrable}<2->
        Un ensemble $X$ est \textbf{dénombrable} s'il existe une injection $i : X \hookrightarrow \N$. On peut alors identifier $X$ à $i(X) \subseteq \N$.
    \end{block}
    \begin{block}{Ensemble parfait}<3->
        Un sous-ensemble $F \subseteq X$, fermé, d'un espace polonais est \textbf{parfait} s'il ne possède pas de points isolés.
    \end{block}
    
\end{frame}

%%%EXAMPLES : Q dénombrable, pas IR, mais IR polonais, un intervalle fermé est parfait et espace de Baire homéomorphe aux irrationnels et $L^p$
%-----------------------------------------------------------

\begin{frame}{Un théorème des mathématiques ordinaires}
    \begin{block}{Théorème (Cantor-Bendixson)}
        Soit $X$, un espace polonais et un fermé $F \subseteq X$ alors il existe une \textbf{ unique décomposition} \begin{align*}
            F = P \cup N,
        \end{align*}
        où $P$ est \textbf{parfait}, $N$ est \textbf{dénombrable} et $P \cap N = \emptyset$.
    \end{block}
\end{frame}


%----------------------------------------------------------------

\section{L'arithmétique du second ordre}

\begin{frame}{Le langage de l'arithmétique du second ordre}
    \begin{align*}
        M = (\alert<2>{|M|, \mathcal{S}_M}, \alert<3>{+_M, \cdot_M, 0_M, 1_M}, \alert<4>{<_M})
    \end{align*}
    \begin{itemize}
        \item<2-> Les \textbf{mots} sont de deux types 
        \begin{itemize}
            \item Les variables numériques ($\N$) ; 
            \item Les variables d'ensembles ($\mathcal{P}(\N)$).
        \end{itemize}
        \item<3-> On construit des \textbf{termes} numériques à l'aide de constantes et de fonctions ;
        \item<4-> Les \textbf{phrases} sont 
        \begin{itemize}
            \item Les formules atomiques $t_1 = t_2$, $t_1 < t_2$, $t_1 \in X$ ;
            \item Les combinaisons et quantifications de celles-ci.
        \end{itemize}
    \end{itemize}
\end{frame}

%----------------------------------------------------------

\begin{frame}{Et quels axiomes ?}
    Ceux que vous connaissez déjà :
    \begin{itemize}
        \item<2-> $\forall n \quad n + 1 \neq 0 ;$
        \item<3-> $\forall m,n \quad m + 1 = n + 1 \implies m = n ;$
        \item<3-> $\dots$
        \item<4-> $(0 \in X \land \forall n \quad \ (n \in X \implies n + 1 \in X) \implies \forall n \quad (n \in X)).$
    \end{itemize}
    \pause \pause \pause \pause Et un axiome de compréhension :
     \begin{align*}
        \exists X \ \forall n \quad (n \in X \Longleftrightarrow \phi(n)).
    \end{align*}

        
\end{frame}



%---------------------------------------------------------------------------

\begin{frame}{Une première restriction ; la compréhension arithmétique }
    \begin{align*}
        \exists X \ (n \in X \Longleftrightarrow \theta(n) ),
    \end{align*}
    où $\theta$ est une formule \textbf{arithmétique}.

\begin{exampleblock}{Exemple}<2->
   On peut parler de l'ensemble des nombres premiers $n \in P \Longleftrightarrow \theta(n) \coloneqq \forall m \forall k \ (n = m\cdot k \longrightarrow m = 1 \lor k = 1) \land n > 1.$
\end{exampleblock}

\end{frame}

%---------------------------------------------------------------------------

\begin{frame}{Arithmétisation}
    \begin{block}{Fonction de couplage}
    Un outil de \textbf{codage} vers notre langage
        \begin{align*}
            p : \ &\N \times \N \hookrightarrow \N \\
            &(m,n) \mapsto (m+n)^2 + m
        \end{align*}
    \end{block}
    \begin{block}{Code pour un espace polonais $\bar{A}$}<2->
        \begin{itemize}
            \item Une ensemble non vide $A \subseteq \N$ ;
            \item Une distance $d : A \times A \to \Q$. ($d \subseteq A \times A \times \Q \subseteq \N$)
        \end{itemize}
    \end{block}
\end{frame}

\begin{frame}{Arithmétisation}
    \begin{block}{Code pour un ouvert de $\bar{A}$}
    Avec notre \textbf{intuition}
        \begin{itemize}
            \item<1-> Un ouvert $U \subseteq \bar{A}$ ;
            \item<2-> $U = \bigcup_{(a,r) \in U^c} B(a,r)$ ;
            \item<3-> $x \in U$.
        \end{itemize}
    Avec notre \textbf{langage} 
        \begin{itemize}
            \item<1-> Un code $U^c \subseteq A \times \Q^+ \subseteq \N$ ;
            \item<2-> $U^c = \{(a,r) \mid a \in U \cap A \text{ et } B(a,r) \subseteq U\}$ ;
            \item<3-> $\lim_n d(x_n,a) < r$ pour un certain $(a,r) \in U^c$.
        \end{itemize}
    \end{block}
    
\end{frame}

%-------------------------------------------------------------------------

\section{Preuve du théorème}

\begin{frame}{Arbre binaires et chemins infinis}
 
    \begin{block}{Arbre binaire}<1->
        On note $2^{<\N} \coloneqq \bigcup_{k \in \N} 2^k$ et on dit que $T \subseteq 2^{<\N}$ est un \textbf{arbre binaire} si chaque segment initial d'un élément de $T$ est encore un élément de $T$.
    \end{block}
    \begin{exampleblock}<2->{L'espace de Cantor $2^{\N}$}
        \begin{itemize}
            \item  L'espace de Cantor est \textbf{polonais}.
            \item<3->  Pour $\tau \in 2^{<\N}$ de longueur $k$, \begin{align*}
            \Omega_{\tau} \coloneqq \{f \in 2^{\N}\ \mid f_{\mid k} = \tau \}.\end{align*}
        \end{itemize}
    \end{exampleblock}
    
\end{frame}

\begin{frame}{Arbre binaires et chemins infinis}
    \begin{block}{Proposition}
        L'ensemble des chemins d'un arbre binaire $T$, noté $[T] \subseteq 2^{\N}$ est un \textbf{fermé} de l'espace de Cantor.
    \end{block}

\end{frame}
\begin{frame}{Petite illustration}
\begin{figure}
    \centering
    \includegraphics[scale=0.5]{arbre imagé.png}
\end{figure}
\end{frame}

\begin{frame}{Arbre binaires et chemins infinis}
    \begin{block}{Proposition}
        L'ensemble des chemins d'un arbre binaire $T$, noté $[T] \subseteq 2^{\N}$ est un \textbf{fermé} de l'espace de Cantor.
    \end{block}

    \begin{block}<2->{Arbre parfait}
        Un arbre binaire $T$ est \textbf{parfait} ssi pour tout $\tau \in T$, il existe deux extensions incompatibles de $\tau$ dans $T$.
    \end{block}
\end{frame}

\begin{frame}{Correspondance entre arbre et ensemble parfait}
    \begin{block}{Lemme du codage par chemins}
        Soit $C \subseteq 2^{\N}$, un \textbf{fermé}, il existe un \textbf{arbre} $T \subseteq 2^{<\N}$ tel que \begin{align*}
        \forall f (f \in C \iff f \text{ est un chemin dans } T).
    \end{align*}
    L'arbre $T$ est parfait ssi $C$ aussi.
    \end{block}
\end{frame}

\begin{frame}{Un axiome plus fort ; compréhension $\Pi_1^1$}
    \begin{align*}
        \exists X \ (n \in X \Longleftrightarrow \phi(n) ),
    \end{align*}
    où $\phi$ est une \textbf{$\Pi_1^1$-formule} i.e. il existe une formule \textbf{arithmétique} $\theta(n,X)$ telle que \begin{align*}
        \phi(n) \Longleftrightarrow \forall Y \ \theta(n,Y).
    \end{align*}
\end{frame}

\begin{frame}{Un théorème équivalent à l'axiome $\Pi_1^1$}
    \begin{block}{Théorème (Cantor-Bendixson pour les arbres binaires)}
        Soit $T \subset 2^{<\N}$, un arbre, le \textbf{noyau parfait} $P$ de $T$ existe. De plus, l'ensemble des chemins de $T$ qui ne sont pas des chemins dans $P$ est \textbf{dénombrable}.
    \end{block}
    \begin{exampleblock}{On utilise bien la compréhension $\Pi_1^1$}<2->
    Le noyau parfait est défini par \begin{align*}
        \tau \in P \Longleftrightarrow \exists \ P' \text{ un sous-arbre parfait non vide de } T_{\tau},
    \end{align*}
    où $T_{\tau}$ est l'ensemble des segments initiaux et des extensions de $\tau$ dans $T$.
\end{exampleblock}
\end{frame}

\begin{frame}{Conclusion}
    \begin{block}{Théorème (Cantor-Bendixson pour l'espace de Cantor)}
        Soit $C \subseteq 2^{\N}$, un \textbf{fermé} de l'espace de Cantor, la compréhension $\Pi_1^1$ prouve qu'il existe une unique décomposition \begin{align*}
            C = P \cup N,
        \end{align*}
        où $P$ est parfait, $N$ est dénombrable et $P \cap N = \emptyset$.
    \end{block}
    \pause 
    Et l'axiome de compréhension $\Pi_1^1$ est le bon système d'axiomes pour le prouver !
\end{frame}

\section{Trailer}

\begin{frame}{The Main Question}
    \begin{block}{Les "Big Five"}
        \begin{itemize}
            \item<4-> $RCA$ (constructivisme) ; Intervalles emboîtés, valeur intermédiaire, $\mathbb{R}$ indénombrable, \dots
            \item<5-> $\dots$
            \item<2-> $ACA$ (prédicativisme) ; Supremum, Boule unité compacte, tout $\Q$-e.v. dénombrable a une base, \dots
            \item<5-> $\dots$
            \item<3-> $\Pi_1^1-CA$ (imprédicativisme) ; Cantor-Bendixson, tout groupe abélien dénombrable est la somme directe d'un divisible et d'un réduit, \dots 
        \end{itemize}
    \end{block}
\end{frame}


\end{document}
