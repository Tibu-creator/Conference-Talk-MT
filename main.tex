\documentclass{beamer} %

%%%BASICS
\usepackage[utf8]{inputenc}
\usepackage{csquotes}
\usepackage{enumerate}
\usepackage{mathtools}
\usepackage{qtree}
\usepackage{tikz}
\usetikzlibrary{trees}
\usetikzlibrary{mindmap}
\usepackage{tikzsymbols}
\usepackage{amsfonts}
\usepackage{hyperref}
\usepackage{tabularx}

\usepackage{multicol}
\usepackage{graphicx}
\usepackage{epigraph}

\renewcommand{\epigraphflush}{center}
\renewcommand{\epigraphwidth}{300}


%%%Commandes

\newtheorem{remark}[theorem]{Remark}

%%%START THEME SETTINGS
\usetheme{Berlin}
\usecolortheme{whale}
\usefonttheme{professionalfonts}
\setbeamertemplate{itemize item}{\color{black} $\blacksquare$}
%%%END THEME SETTINGS

%%%START APA
\usepackage[british]{babel}
\usepackage[backend=biber,style=apa]{biblatex}
\DeclareLanguageMapping{british}{british-apa}
\addbibresource{references.bib}
%% APA citing
%% \cite{t} - Uthor und Richter, 2010
%% \textcite{t} - Uthor und Riter (2010)
%% \parencite{t} - (Uthor & Riter, 2010)
%% \parencite[Chapt.~4]{t} - (Uthor & Riter, 2010, S. 15)
%%%END APA


\title[Determinacy axioms]{Determinacy in and out second order arithmetic}
\subtitle[]{An introduction to the proof theoretic strength of the determinacy scale}
\institute[Proof Theory Conference]{Proof Theory Conference \and UCLouvain}
\author{Thibaut Kouptchinsky}

\date{December 20, 2022}

\begin{document}

\begin{frame}
	\titlepage
\end{frame}

%------------------------------------------------------

\begin{frame}{Program}
    \tableofcontents
\end{frame}
% Presentation structure

%--------------------------------------------------------
\section{A tool of descriptive set theory}

\begin{frame}{What is determinacy?}
    
    \begin{tabular}{llllllllll}
        \multicolumn{10}{l}{Consider a set $A$ and a payoff set $X \subseteq A^{\omega}$.}\\
        &&&&&&&&&\\
        I:\ & $a_0$ &       & $a_2$ &       &          &$a_{2n}$   &           & & \\
            &       &       &       &       & $\cdots$ &           &           & $\cdots$ & \qquad ${(a_i)}_{i < \omega} \overset{?}{\in} X$\\
        II:\ &       & $a_1$ &       & $a_3$ &          &           &$a_{2n+1}$ & & \\
        &&&&&&&&&\\
        \multicolumn{10}{l}{Player I wins if yes. Otherwise player II wins.} \\ 
        &&&&&&&&&\\
    \end{tabular}

    Axiom of determinacy ($\mathsf{AD}$): ``All these games are determined''.
    
    (False in $\mathsf{ZF + C}$.)
\end{frame}

%----------------------------------------------------------------

\begin{frame}{Motivations and applications}

    \begin{theorem}[Mycielski-Swierczkowski; Mazur, Banach; Davis]
        $\mathsf{ZF + AD}$ proves  that every set of real numbers is Lebesgue measurable (M1), has 
        the Baire property (M2), and has the perfect set property (M3).
    \end{theorem}

    \begin{itemize}
        \item<2-> Study these properties for projective $\Sigma^1_n$ sets in $\omega^{\omega}$
        (Blackwell, 1967).
        \item<3-> Are $\Sigma^1_2$, $\Sigma^1_3$, etc sets Lebesgue measurable?
        \item<4> Applications in measure theory, descriptive set theory, harmonic analysis, ergodic theory, 
        dynamical systems etc.
    \end{itemize}

\end{frame}

%----------------------------------------------------------------

\begin{frame}{Borel Determinacy}
    \begin{itemize}
        \item<1-> First best result (1964): $\mathsf{Det(\Sigma^0_3)}$ by Davis.
        \item<2-> The proof can be carried out in $\mathsf{ZC^- + \Sigma_1 Replacement}$ (Martin).
        \item<3-> Friedman (1968): Borel determinacy requires existence of $V_{\omega_1}$. 
    \end{itemize}

    \begin{theorem}[Martin, $\mathsf{ZFC}$]<4>
        All Borel games are determined.
    \end{theorem}

\end{frame}

%----------------------------------------------------------------

\section{The theorem of Wolfe as a warm up}

\begin{frame}
    \begin{theorem}[$\mathsf{ZC}^- + \Sigma_1$-\textsc{Replacement}]
        All $\Sigma^0_2$ games are determined.\label{Sigma2Det}
    \end{theorem}
\end{frame}

%----------------------------------------------------------

\begin{frame}
    \begin{lemma}
        Let $B \subseteq A \subseteq [T]$ with $B$ being closed. If player I has no winning strategy in the 
        game $G(T,A)$, then there is a strategy $\tau$ for II such that every $x \in [\tau]$ has a finite 
        initial segment $p$ verifying \begin{align*}
            [T_p] \cap B = \emptyset \qquad \text{and} \qquad \text{I has still no winning strategy in }G(T_p,A) 
        \end{align*} 
    \end{lemma}
\end{frame}

%----------------------------------------------------------

\begin{frame}{Proof of the theorem of Wolfe}
    
\end{frame}

%----------------------------------------------------------


\section{Determinacy of $\Pi^0_3$ Differences}
\begin{frame}{} 
    \begin{theorem}[Montalb\'an and Shore]
        $\forall n \in \omega$, $\Pi^1_{n+2}$-$\mathsf{CA}_0 \vdash \mathsf{Det}(n$-$\Pi^0_3)$
    \end{theorem}
    \pause
    However, $\Pi^1_{n+2}$-$\mathsf{CA}_0$ is not the right set of axioms for 
    $\mathsf{Det}(n$-$\Pi^0_3)$.

    \begin{theorem}[MedSalem and Tanaka]<3->
        Borel determinacy does not imply $\Delta^1_2$-$\mathsf{CA}_0$.
    \end{theorem}
\end{frame}

%----------------------------------------------------------

\begin{frame}
    \begin{definition}
        We define $\Sigma_{|s|+2}^1$ relations $P^s(S)$ by induction on $|s| \leq m$:
        \begin{itemize}
            \item When $|s|=0$, $P^{\emptyset}(S)$ iff \begin{align}
                &\text{I (II) has a winning strategy in } G(A,S) \text{ if } l \text{ is even (odd)}.\label{base}
            \end{align}
            \item For $|s|=n+1$ and $l$ is even, $P^{s}(S)$ iff there is a quasistrategy $U$ for I in $S$ such 
            that \begin{align}
                [U] \subseteq A \cup A_{l, s(n)} \qquad \text{and} \qquad 
                P^{s[n]}(U) \text{ fails}.\label{even}
            \end{align}
            \item For $|s|=n+1$ and $l$ is odd, $P^{s}(S)$ iff there is a quasistrategy $U$ for II in $S$ such 
            that \begin{align}
                [U] \subseteq \Bar{A} \cup A_{l, s(n)} \qquad \text{and} \qquad
                P^{s[n]}(U) \text{ fails}.\label{odd}
            \end{align}  
        \end{itemize}
        A quasistrategy $U$ witnesses $P^{s}(S)$ if $U$ is as required in the appropriate clause, the latter being 
        a $\Pi_{|s|+1}^1$ sentence.\label{P}
    \end{definition}
\end{frame}

%------------------------------------------------------------------

\begin{frame}
    \begin{definition}
        A quasistrategy $U$ locally witnesses $P^s(S)$ if $|s| = n+1$ and $U$ is a quasistrategy for I (II) if 
        $l$ is even (odd) and there is $D \subseteq S$ such that, for every $d \in D$, there is a quasistrategy 
        $R^d$ for II (I) if $l$ is even (odd) in $S_d$ such that the following conditions are satisfied:
        \begin{enumerate}
            \item $\forall d \in D\cap U$, \ $U_d \cap R^d$ witnesses $P^s(R^d)$.
            \item $[U] \setminus \bigcup_{d \in D} [R^d] \subseteq A \ (\text{resp. }\Bar{A})$.
            \item $\forall p \in S \ \exists^{\leq 1} d \in D, \ d \subseteq p \land p \in R^d$.
        \end{enumerate}
        We observe that ``$U$ locally witnesses $P^s(S)$'' is a $\Sigma_{|s|+2}^1$ sentence.
    \end{definition}
\end{frame}

%----------------------------------------------------------------

\begin{frame}
    \begin{lemma}[1]
        If $U$ locally witnesses $P^s(S)$, then $U$ witnesses $P^s(S)$.\label{nolocal}
    \end{lemma}
\end{frame}

%----------------------------------------------------------------

\begin{frame}{Proof of lemma 1}
    
\end{frame}

%----------------------------------------------------------

\begin{frame}

    \begin{definition}
        We say that $P^s(S)$ fails everywhere if $P^s(S_p)$ fails for every $p \in S$. This is a $\Pi^1_{|s|+2}$ 
        sentence.
    \end{definition}
    \begin{lemma}[2]
        If $P^s(S)$ fails, then there is a quasistrategy $W$ for I (II) if $l$ is odd (even) in $S$ such that 
        $P^s(W)$ fails everywhere.\label{failure}
    \end{lemma}
\end{frame}

%----------------------------------------------------------------

\begin{frame}{Proof of lemma 2}
    
\end{frame}

%----------------------------------------------------------

\begin{frame}
    \begin{definition}
        For $|s| = n+1$, $W$ strongly witnesses $P^s(S)$ if, for all $p \in W$, $W_p$ witnesses $P^s(S_p)$, that 
        is, $W$ witnesses $P^s(S)$ and $P^{s[n]}(W)$ fails everywhere. This is a $\Pi_{|s+1|^1}$ 
        sentence.
    \end{definition}
\end{frame}

%----------------------------------------------------------------

\begin{frame} 
    \begin{lemma}[3]
        If $P^s(S)$, then there is a $W$ that strongly witnesses it.\label{enpowerment}
    \end{lemma}
\end{frame}

%----------------------------------------------------------------

\begin{frame}{Proof of lemma 3}
    
\end{frame}

%----------------------------------------------------------

\begin{frame}
    \begin{lemma}[4]
        If $|s| = n+1$, then at least one of $P^s(S)$ and $P^{s[n]}(S)$ holds.\label{binary}
    \end{lemma}
\end{frame}

%----------------------------------------------------------------

\begin{frame}{Proof of lemma 4}
    
\end{frame}

%----------------------------------------------------------

\begin{frame}{Proof of the theorem}
    
\end{frame}

\section*{}
\begin{frame}
    \huge{Thank you for your attention!}
\end{frame}
\end{document}
