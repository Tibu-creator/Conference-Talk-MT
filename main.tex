\documentclass{beamer} %

%%%BASICS
\usepackage[utf8]{inputenc}
\usepackage{csquotes}
\usepackage{enumerate}
\usepackage{mathtools}
\usepackage{qtree}
\usepackage{tikz}
\usetikzlibrary{trees}
\usepackage{amsfonts}
\usepackage{hyperref}
\usepackage{tabularx}

\usepackage{multicol}
\usepackage{graphicx}
\usepackage{epigraph}

\renewcommand{\epigraphflush}{center}
\renewcommand{\epigraphwidth}{300}


%%%Commandes

\newtheorem{remark}[theorem]{Remark}

%%%START THEME SETTINGS
\usetheme{Berlin}
\usecolortheme{whale}
\usefonttheme{professionalfonts}
\setbeamertemplate{itemize item}{\color{black} $\blacksquare$}
%%%END THEME SETTINGS

%%%START APA
\usepackage[british]{babel}
\usepackage[backend=biber,style=apa]{biblatex}
\DeclareLanguageMapping{british}{british-apa}
\addbibresource{references.bib}
%% APA citing
%% \cite{t} - Uthor und Richter, 2010
%% \textcite{t} - Uthor und Riter (2010)
%% \parencite{t} - (Uthor & Riter, 2010)
%% \parencite[Chapt.~4]{t} - (Uthor & Riter, 2010, S. 15)
%%%END APA


\title[]{Determinacy in and out second order arithmetic}
\subtitle[]{An introduction to the proof theoretic strength of the determinacy scale}
\institute[Proof Theory Conference]{}
\author{Thibaut Kouptchinsky}

\date{December 22, 2022}

\begin{document}

\begin{frame}
	\titlepage
\end{frame}

%------------------------------------------------------

\begin{frame}{Program}
    \tableofcontents
\end{frame}
% Presentation structure

%--------------------------------------------------------

\section{Who am I?}

\begin{frame}
    
\end{frame}


%
\section{Introduction}

\begin{frame}{What is determinacy?}
    
    \begin{tabular}{llllllllll}
        \multicolumn{10}{l}{Consider a set $A$ and a payoff set $X \subseteq A^{\omega}$.}\\
        &&&&&&&&&\\
        I:\ & $a_0$ &       & $a_2$ &       &          &$a_{2n}$   &           & & \\
            &       &       &       &       & $\cdots$ &           &           & $\cdots$ & \qquad ${(a_i)}_{i < \omega} \overset{?}{\in} X$\\
        II:\ &       & $a_1$ &       & $a_3$ &          &           &$a_{2n+1}$ & & \\
        &&&&&&&&&\\
        \multicolumn{10}{l}{Player I wins if yes. Otherwise player II wins.} \\ 
        &&&&&&&&&\\
    \end{tabular}

    Axiom of determinacy ($\mathsf{AD}$): ``All these games are determined''.
    
    (False in $\mathsf{ZF + C}$.)
\end{frame}

%----------------------------------------------------------------

\begin{frame}{Motivations and applications}

    \begin{theorem}[Mycielski-Swierczkowski; Mazur, Banach; Davis]
        $\mathsf{ZF + AD}$ proves  that every sets of real numbers is Lebesgue measurable (M1), has 
        the Baire property (M2), and has the perfect set property (M3).
    \end{theorem}

    \begin{itemize}
        \item<2-> Study these properties for projective $\Sigma^1_n$ sets in $\omega^{\omega}$. 
        \item<3-> Are $\Sigma^1_2$, $\Sigma^1_3$, etc sets Lebesgue measurable? 
        \item<4> Applications in measure theory, descriptive set theory, harmonic analysis, ergodic theory, 
        dynamical systems etc.
    \end{itemize}

\end{frame}

%----------------------------------------------------------------

\begin{frame}{Borel Determinacy}
    \begin{itemize}
        \item<1-> First best result (1964): $\mathsf{Det(\Sigma^0_3)}$ by Davis.
        \item<2-> The proof can be carried out in $\mathsf{ZC^- + \Sigma_1 Replacement}$.
        \item<3-> Friedman (1968): Borel determinacy requires existence of $V_{\omega_1}$. 
    \end{itemize}

    \begin{theorem}[Martin, $\mathsf{ZFC}$]<4>
        All Borel games are determined.
    \end{theorem}

\end{frame}

%----------------------------------------------------------------

\begin{frame}{Sketch of the proof}
    
\end{frame}

%----------------------------------------------------------------

\section{Inside second order arithmetic}

\begin{frame}{A conservation result}
    \begin{theorem}
        $\mathsf{ZFC}^-$ is a $\Pi^1_4$ conservative extension of $\mathsf{Z}_2$.    
    \end{theorem}
    
    \begin{itemize}
        \item<2-> We can construct $\mathsf{L(X)}$ in $\mathsf{Z}_2$;
        \item<3-> We can show in that $\mathsf{Z}_2$ that $\mathsf{L(X)} \models \mathsf{ZFC}^-$;
        \item<4-> We use Shoenfield Absoluteness theorem ($\mathsf{\Pi_1^1}-CA_0$).
    \end{itemize}

\end{frame}

%----------------------------------------------------------

\begin{frame}{Some right axioms systems}
    \begin{theorem}[Steel, Simpson]
        Over $\mathsf{RCA}_0$, $\mathsf{Det(\Sigma_1^0)}$ is equivalent to $\mathsf{ATR}_0$. 
    \end{theorem}

    \begin{theorem}[Tanaka]
        $\mathsf{Det(\Sigma_2^0)}$ is equivalent to $\mathsf{\Sigma_1^1-MI}$.
    \end{theorem}
\end{frame}

%----------------------------------------------------------

\begin{frame}{How much determinacy can we prove in $\mathsf{Z}_2$?} 
    \begin{theorem}[Montalb\'an and Shore]
        $\Pi^1_{n+2}$-$\mathsf{CA}_0 \vdash \mathsf{Det}(n$-$\Pi^0_3)$ but
        $\Delta^1_{n+2}$-$\mathsf{CA}_0 \not\vdash \mathsf{Det}(n$-$\Pi^0_3)$.
    \end{theorem}
    However, $\Pi^1_{n+2}$-$\mathsf{CA}_0$ is not the right set of axioms for 
    $\mathsf{Det}(n$-$\Pi^0_3)$.

    \begin{theorem}[MedSalem and Tanaka]
        Borel determinacy does not imply $\Delta^1_2$-$\mathsf{CA}_0$.
    \end{theorem}
\end{frame}

%----------------------------------------------------------

\begin{frame}{Reversals}
 
    \begin{theorem}[Hachtman]
        $\mathsf{Det(\Sigma_3^0)}$ is equivalent to the existence of a $\beta-$model 
        satisfying $\mathsf{\Pi_1^2-MI}$.
    \end{theorem}

    \begin{theorem}[Aguilera and Welch]
        Over $\Pi_1^1$-$\mathsf{CA}_0$, for each $m\in \mathbb{N}$ we have an 
        equivalence between 
        \begin{enumerate}
            \item $\mathsf{Det}(m$-$\Pi_3^0)$,
            \item Every real belongs to a $\beta$-model of $\Pi_{m+1}^1$-$\mathsf{MI}$.
        \end{enumerate}
    \end{theorem}
\end{frame}

%------------------------------------------------------------------

\section{Back in \texorpdfstring{$\mathsf{ZF}$}{ZF} set theory}

\begin{frame}{Game encoding models}
    \begin{theorem}[Friedman $<$ Martin $<$ M. and S.]
        We cannot prove $\mathsf{Det}(\Sigma_0^5)<\mathsf{Det}(\Sigma_0^4)
        <\mathsf{Det}(\omega$-$\Pi_0^3)$ in $\mathsf{ZCF^-}$ 
    \end{theorem}

    \begin{itemize}
        \item<2-> Same technique as for the limitative result of Friedman for 
            Borel determinacy.
        \item<3-> The games are deemed to encode fragments of set theory using Gödel numbering.
    \end{itemize}
\end{frame}

%----------------------------------------------------------------

\begin{frame}{Sketch of the proof for $\mathsf{Det}(\Sigma_0^4)$}
    
\end{frame}

%----------------------------------------------------------------

\begin{frame}{Measurability properties}

    \begin{theorem}[Kechris, Martin]
        In $\mathsf{ZF + AC}_{\omega}(\omega^{\omega})$, $\mathsf{Det}(\Pi^1_n)$ proves that 
        every $\Sigma^1_{n+1}$ sets of reals satisfies M1, M2 and M3.
    \end{theorem}

    \begin{theorem}[Shelah-Woodin]
        Given $n \in \omega$, if there are $n$ Woodin cardinals with a measurable cardinal above 
        them, then every $\Sigma^1_{n+2}$ sets of reals satisfies M1, M2 and M3.
    \end{theorem}
\end{frame}

%----------------------------------------------------------------

\begin{frame}{Determinacy and high cardinal hypotheses}
    \begin{theorem}
        Given $n \in \omega$, if there are $n$ Woodin cardinals with a measurable cardinal above 
        them, then $\mathsf{Det}(\Pi^1_{n+1})$.
    \end{theorem}

    \begin{remark}
        This is a corollary from a theorem of Martin-Steel, which is out of the scope of the 
        present talk.
    \end{remark} 
\end{frame}

%----------------------------------------------------------------

\section{Perspectives and Material}

\begin{frame}
    
\end{frame}

%----------------------------------------------------------------

\end{document}
