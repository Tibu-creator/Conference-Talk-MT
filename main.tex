\documentclass{beamer} %

%%%BASICS
\usepackage[utf8]{inputenc}
\usepackage{csquotes}
\usepackage{enumerate}
\usepackage{mathtools}
\usepackage{qtree}
\usepackage{tikz}
\usetikzlibrary{trees}
\usepackage{amsfonts}
\usepackage{hyperref}
\usepackage{tabularx}

\usepackage{multicol}
\usepackage{graphicx}
\usepackage{epigraph}

\renewcommand{\epigraphflush}{center}
\renewcommand{\epigraphwidth}{300}


%%%Commandes

\newcommand{\N}{\mathbb{N}}
\newcommand{\Q}{\mathbb{Q}}

%%%START THEME SETTINGS
\usetheme{Berlin}
\usecolortheme{whale}
\usefonttheme{professionalfonts}
\setbeamertemplate{itemize item}{\color{black} $\blacksquare$}
%%%END THEME SETTINGS

%%%START APA
\usepackage[british]{babel}
\usepackage[backend=biber,style=apa]{biblatex}
\DeclareLanguageMapping{british}{british-apa}
\addbibresource{references.bib}
%% APA citing
%% \cite{t} - Uthor und Richter, 2010
%% \textcite{t} - Uthor und Riter (2010)
%% \parencite{t} - (Uthor & Riter, 2010)
%% \parencite[Chapt.~4]{t} - (Uthor & Riter, 2010, S. 15)
%%%END APA


\title[]{Determinacy in and out second order arithmetic}
\subtitle[]{An introduction to the proof theoretic strength of the determinacy scale}
\institute[Proof Theory Conference]{}
\author{Thibaut Kouptchinsky}

\date{December 22, 2022}

\begin{document}

\begin{frame}
	\titlepage
\end{frame}

%------------------------------------------------------

\begin{frame}{Program}
    \tableofcontents
\end{frame}
% Presentation structure

%--------------------------------------------------------

\section{Who am I?}

\begin{frame}
    
\end{frame}


%
\section{Introduction}

\begin{frame}{What is determinacy?}
    
    \begin{tabular}{llllllllll}
        \multicolumn{10}{l}{Consider a set $A$ and a payoff set $X \subseteq A^{\omega}$.}\\
        &&&&&&&&&\\
        I:\ & $a_0$ &       & $a_2$ &       &          &$a_{2n}$   &           & & \\
            &       &       &       &       & $\cdots$ &           &           & $\cdots$ & \qquad ${(a_i)}_{i < \omega} \overset{?}{\in} X$\\
        II:\ &       & $a_1$ &       & $a_3$ &          &           &$a_{2n+1}$ & & \\
        &&&&&&&&&\\
        \multicolumn{10}{l}{Player I wins if yes. Otherwise player II wins.} \\ 
        &&&&&&&&&\\
    \end{tabular}

    Axiom of determinacy ($\mathsf{AD}$): ``All these games are determined''.
    
    (False in $\mathsf{ZF + C}$.)
\end{frame}

%----------------------------------------------------------------

\begin{frame}{Motivations and applications}

    \begin{theorem}[Mycielski-Swierczkowski; Mazur, Banach; Davis]
        $\mathsf{ZF + AD}$ proves  that every sets of real numbers is Lebesgue measurable, has 
        the Baire property, and has the perfect set property.
    \end{theorem}

    Study these properties for projective $\Sigma^1_n$ sets in $\omega^{\omega}$. 

    \begin{center}
        Are $\Sigma^1_3$ sets Lebesgue measurable? 
        
        Kechris and Martin: ``Yes, provided $\mathsf{Det(\Pi^1_2)\ (+ AC_{\omega}(\omega^{\omega}))}$''.
    \end{center}

    Applications in measure theory, descriptive set theory, harmonic analysis, ergodic theory, 
    dynamical systems etc.


\end{frame}

%----------------------------------------------------------------

\section{Inside second order arithmetic}

\begin{frame}{}
    
\end{frame}

%----------------------------------------------------------

\section{Outside}

\begin{frame}{}

\end{frame}

%----------------------------------------------------------------

\end{document}
