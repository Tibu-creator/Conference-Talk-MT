\documentclass{beamer} %

%%%BASICS
\usepackage[utf8]{inputenc}
\usepackage{csquotes}
\usepackage{enumerate}
\usepackage{mathtools}
\usepackage{qtree}
\usepackage{tikz}
\usetikzlibrary{trees}
\usetikzlibrary{mindmap}
\usepackage{tikzsymbols}
\usepackage{amsfonts}
\usepackage{hyperref}
\usepackage{tabularx}

\usepackage{multicol}
\usepackage{graphicx}
\usepackage{epigraph}

\renewcommand{\epigraphflush}{center}
\renewcommand{\epigraphwidth}{300}


%%%Commandes

\newtheorem{remark}[theorem]{Remark}

%%%START THEME SETTINGS
\usetheme{Berlin}
\usecolortheme{whale}
\usefonttheme{professionalfonts}
\setbeamertemplate{itemize item}{\color{black} $\blacksquare$}
%%%END THEME SETTINGS

%%%START APA
\usepackage[british]{babel}
\usepackage[backend=biber,style=apa]{biblatex}
\DeclareLanguageMapping{british}{british-apa}
\addbibresource{references.bib}
%% APA citing
%% \cite{t} - Uthor und Richter, 2010
%% \textcite{t} - Uthor und Riter (2010)
%% \parencite{t} - (Uthor & Riter, 2010)
%% \parencite[Chapt.~4]{t} - (Uthor & Riter, 2010, S. 15)
%%%END APA


\title[Determinacy axioms]{Determinacy in and out second order arithmetic}
\subtitle[]{An introduction to the proof theoretic strength of the determinacy scale}
\institute[Proof Theory Conference]{Proof Theory Conference \and UCLouvain}
\author{Thibaut Kouptchinsky}

\date{December 20, 2022}

\begin{document}

\begin{frame}
	\titlepage
\end{frame}

%------------------------------------------------------

\begin{frame}{Program}
    \tableofcontents
\end{frame}
% Presentation structure

%--------------------------------------------------------
\section{A tool of descriptive set theory}

\begin{frame}{What is determinacy?}
    
    \begin{tabular}{llllllllll}
        \multicolumn{10}{l}{Consider a set $A$ and a payoff set $X \subseteq A^{\omega}$.}\\
        &&&&&&&&&\\
        I:\ & $a_0$ &       & $a_2$ &       &          &$a_{2n}$   &           & & \\
            &       &       &       &       & $\cdots$ &           &           & $\cdots$ & \qquad ${(a_i)}_{i < \omega} \overset{?}{\in} X$\\
        II:\ &       & $a_1$ &       & $a_3$ &          &           &$a_{2n+1}$ & & \\
        &&&&&&&&&\\
        \multicolumn{10}{l}{Player I wins if yes. Otherwise player II wins.} \\ 
        &&&&&&&&&\\
    \end{tabular}

    Axiom of determinacy ($\mathsf{AD}$): ``All these games are determined''.
    
    (False in $\mathsf{ZF + C}$.)
\end{frame}

%----------------------------------------------------------------

\begin{frame}{The Borel and projective hierarchy}
    
\end{frame}

%----------------------------------------------------------------

\begin{frame}{Motivations and applications}

    \begin{theorem}[Mycielski-Swierczkowski; Mazur, Banach; Davis]
        $\mathsf{ZF + AD}$ proves  that every set of real numbers is Lebesgue measurable (M1), has 
        the Baire property (M2), and has the perfect set property (M3).
    \end{theorem}

    \begin{itemize}
        \item<2-> Study these properties for projective $\Sigma^1_n$ sets in $\omega^{\omega}$
        (Blackwell, 1967).
        \item<3-> Are $\Sigma^1_2$, $\Sigma^1_3$, etc sets Lebesgue measurable?
        \item<4> Applications in measure theory, descriptive set theory, harmonic analysis, ergodic theory, 
        dynamical systems etc.
    \end{itemize}

\end{frame}

%----------------------------------------------------------------

\begin{frame}{Borel Determinacy}
    \begin{itemize}
        \item<1-> First best result (1964): $\mathsf{Det(\Sigma^0_3)}$ by Davis.
        \item<2-> The proof can be carried out in $\mathsf{ZC^- + \Sigma_1 Replacement}$ (Martin).
        \item<3-> Friedman (1968): Borel determinacy requires existence of $V_{\omega_1}$. 
    \end{itemize}

    \begin{theorem}[Martin, $\mathsf{ZFC}$]<4>
        All Borel games are determined.
    \end{theorem}

\end{frame}

%----------------------------------------------------------------

\section{The theorem of Wolfe as a warm up}

\begin{frame}{The theorem of Wolfe}
    \begin{theorem}[$\mathsf{ZC}^- + \Sigma_1$-\textsc{Replacement}]
        All $\Sigma^0_2$ games are determined.\label{Sigma2Det}
    \end{theorem}
\end{frame}

%----------------------------------------------------------

\begin{frame}
    \begin{definition}[Player I's strategies]
        Given $G(T, X)$, a strategy $S_{I}$ is a subtree of $T$ such that \begin{enumerate}
            \item<2-> Every even-length node has one unique child,
            \item<3-> Every child of an odd-length node that lies in $T$ lies also in the strategy,
        \end{enumerate}
        \pause
        \pause
        \pause
        A strategy $S$ is said to be winning if $[S] \subseteq X$.
    \end{definition}
    \pause 
    In a quasistrategy the player's response has not to be unique.
\end{frame}

%----------------------------------------------------------

\begin{frame}{Example: A binary game}
    \scalebox{0.75}
    {
    \begin{tikzpicture}
        \begin{scope}[shift={(-10,10)}]
                    \Tree [.{$\langle \ \rangle$} 
                        [.{$\langle 0 \rangle$} [.{$\langle 00 \rangle$} 
                            [.{$\langle 000 \rangle$} {$\cdots$} {$\cdots$} ] 
                            [.{$\langle 001 \rangle$} {$\cdots$} {$\cdots$} ] ] 
                        [.{$\langle 01 \rangle$} 
                            [.{$\langle 010 \rangle$} {$\cdots$} {$\cdots$} ] 
                            [.{$\langle 011 \rangle$} {$\cdots$} {$\cdots$} ] ] ] 
                        [.{$\langle 1 \rangle$} [.{$\langle 10 \rangle$} 
                            [.{$\langle 100 \rangle$} {$\cdots$} {$\cdots$} ] 
                            [.{$\langle 101 \rangle$} {$\cdots$} {$\cdots$} ] ]
                        [.{$\langle 11 \rangle$} 
                            [.{$\langle 110 \rangle$} {$\cdots$} {$\cdots$} ] 
                            [.{$\langle 111 \rangle$} {$\cdots$} {$\cdots$} ] ] ].{$\langle 1 \rangle$} ]
                \end{scope}
        \end{tikzpicture}
        }
        \phantom{a}

        \phantom{a}

        \phantom{a}

        \phantom{a}

        \phantom{a}

        \phantom{a}

        \phantom{a}

        \phantom{a}

        \phantom{a}

        \phantom{a}
\end{frame}

%----------------------------------------------------------

\begin{frame}{Strategy and quasistrategy}
    \scalebox{0.75}{
        \begin{tikzpicture}[]
            \begin{scope}[]
                \Tree [.{$\langle \ \rangle$} [.{$\langle 0 \rangle$} [.{$\langle 00 \rangle$} 
                [.{$\langle 000 \rangle$} [
                    [.{$\langle 0000 \rangle$} {$\cdots$} {$\cdots$} ]
                    [.{$\langle 0001 \rangle$} {$\cdots$} {$\cdots$} ] ] ] 
                [.{$\langle 001 \rangle$} [
                    [.{$\langle 0010 \rangle$} {$\cdots$} {$\cdots$} ]
                    [.{$\langle 0011 \rangle$} {$\cdots$} {$\cdots$} ] ] ] ] 
                [.{$\langle 01 \rangle$} 
                [.{$\langle 010 \rangle$} [
                    [.{$\langle 0100 \rangle$} {$\cdots$} {$\cdots$} ]
                    [.{$\langle 0101 \rangle$} {$\cdots$} {$\cdots$} ] ] ]
                [.{$\langle 011 \rangle$} [
                    [.{$\langle 0110 \rangle$} {$\cdots$} {$\cdots$} ]
                    [.{$\langle 0111 \rangle$} {$\cdots$} {$\cdots$} ] ] ] ] 
                ] ]
            \end{scope}
        \end{tikzpicture}
    }
        
        \phantom{a}

        \phantom{a}

        \phantom{a}

        \phantom{a}

        \phantom{a}

        \phantom{a}

        \phantom{a}

        \phantom{a}

        \phantom{a}

        \phantom{a}
\end{frame}

%----------------------------------------------------------

\begin{frame}{A technical lemma}
    \begin{lemma}
        Let $B \subseteq A \subseteq [T]$ with $B$ being closed. If player I has no winning strategy in the 
        game $G(T,A)$, then there is a strategy $\tau$ for II such that every $x \in [\tau]$ has a finite 
        initial segment $p$ verifying \begin{align*}
            [T_p] \cap B = \emptyset \qquad \text{and} \qquad \text{I has still no winning strategy in }G(T_p,A) 
        \end{align*} 
    \end{lemma}
\end{frame}

%----------------------------------------------------------

\begin{frame}{Proof of the theorem of Wolfe}
    \begin{align*}
        A = \bigcup_{i < \omega} A_i, \qquad A_i \text{ closed in } [T];
    \end{align*}
    \begin{itemize}
        \item<2-> Suppose $G(T,A)$ is not a win for I\@;
        \item<3-> Apply the lemma for $B = A_0$ to get $\tau_0 \subset T$ and $p_0$;
        \item<4-> Apply the lemma for $B = A_{n+1}$ to get $\tau_{n+1} \subset T_{p_{n}}$ and $p_{n+1}$;
        \item<5-> \dots
        \item<6-> Our $\tau$ avoid all the $A_i$ and hence is winning for II\@.
    \end{itemize}
\end{frame}

%----------------------------------------------------------


\section{Determinacy of \texorpdfstring{$\Pi^0_3$}{Pi03} Differences}
\begin{frame}{} 
    \begin{theorem}[Montalb\'an and Shore]
        $\forall n \in \omega$, $\Pi^1_{n+2}$-$\mathsf{CA}_0 \vdash \mathsf{Det}(n$-$\Pi^0_3)$
    \end{theorem}
    \pause
    However, $\Pi^1_{n+2}$-$\mathsf{CA}_0$ is not the right set of axioms for 
    $\mathsf{Det}(n$-$\Pi^0_3)$.

    \begin{theorem}[MedSalem and Tanaka]<3->
        Borel determinacy does not imply $\Delta^1_2$-$\mathsf{CA}_0$.
    \end{theorem}
    \pause
    \pause
    $\mathsf{Det}(n$-$\Pi^0_3)$ is the $\Pi^1_3$ sentence
    \begin{align*}
        \forall X \ \exists Y \ \forall Z \ (X \in n\text{-}\Pi^0_3) \rightarrow \begin{cases}
            Y \in S_I \land Z \in S_{II} \rightarrow Y \bigoplus Z \in X;
            \\  Y \in S_{II} \land Z \in S_{I} \rightarrow Z \bigoplus Y \in X.
        \end{cases}
    \end{align*}
\end{frame}

%------------------------------------------------------------------

\begin{frame}
    Consider $\emptyset = A_m \subseteq \dots \subseteq A_1 \subseteq A_0$, $\Pi_0^3$ sets.

    \begin{align*}
        A_i = \bigcap_{k < \omega} A_{i,k} \qquad \text{ and } \qquad A_{i,k} = \bigcup_{j < \omega} A_{i, k, j}.
    \end{align*}
\end{frame}
%----------------------------------------------------------

\begin{frame}
    \begin{definition}
        Assume $0 \leq k < \omega$.
        \begin{enumerate}
            \item<2-> $\Sigma^1_k$-$\mathsf{DC}_0$: \begin{align*}
                \forall n \ \forall X \ \exists Y \ \eta(n, X, Y) \rightarrow \exists Z \ \forall n \ \eta(n, {(Z)}^n, {(Z)}_n). 
            \end{align*}
            \item<3-> Strong $\Sigma^1_k$-$\mathsf{DC}_0$: \begin{align*}
                \exists Z \ \forall n \ \forall Y \ (\eta(n, {(Z)}^n, Y) \rightarrow \eta(n, {(Z)}^n, {(Z)}_n)).
            \end{align*}
        \end{enumerate}
    \end{definition}
    \pause
    \pause
    \pause
    Strong $\Sigma^1_{m+2}$-$\mathsf{DC}_0$ is $\Pi^1_4$ conservative over $\Pi^1_{m+2}$-$\mathsf{CA}_0$.
\end{frame}

%----------------------------------------------------------

\begin{frame}
    \begin{definition}
        We define $\Sigma_{|s|+2}^1$ relations $P^s(S)$ by induction on $|s| \leq m$:
        \begin{itemize}
            \item<2-> When $|s|=0$, $P^{\emptyset}(S)$ iff \begin{align*}
                &G(A,S) \text{ is a win for I if } l \coloneq m-|s| \text{ is even }.
            \end{align*}
            \item<3-> For $|s|=n+1$ and $l$ is even, $P^{s}(S)$ iff there is a quasistrategy $U$ for I in $S$ such 
            that \begin{align*}
                [U] \subseteq A \cup A_{l, s(n)} \qquad \text{and} \qquad 
                P^{s[n]}(U) \text{ fails}.
            \end{align*}
        \end{itemize}
    \end{definition}
    \pause 
    \pause
    \pause
    A quasistrategy $U$ witnesses $P^{s}(S)$ if $U$ is as required in the appropriate clause, the latter being 
    a $\Pi_{|s|+1}^1$ sentence.
\end{frame}

%------------------------------------------------------------------

\begin{frame}
    \begin{definition}
        A quasistrategy $U$ for I locally witnesses $P^s(S)$ if $|s| = n+1$ and 
        $l$ is even if:  
        $\exists D \subseteq S \ \forall d \in D$, there is a quasistrategy 
        $R^d$ for II in $S_d$ such that:
        \begin{enumerate}
            \item<2-> $\forall d \in D\cap U$, \ $U_d \cap R^d$ witnesses $P^s(R^d)$.
            \item<3-> $[U] \setminus \bigcup_{d \in D} [R^d] \subseteq A$.
            \item<4-> $\forall p \in S \ \exists^{\leq 1} d \in D, \ d \subseteq p \land p \in R^d$.
        \end{enumerate}
        \pause
        \pause
        \pause
        \pause
        We observe that ``$U$ locally witnesses $P^s(S)$'' is a $\Sigma_{|s|+2}^1$ sentence.
    \end{definition}
\end{frame}

%----------------------------------------------------------------

\begin{frame}
    \begin{lemma}[1]
        If $U$ locally witnesses $P^s(S)$, then $U$ witnesses $P^s(S)$.\label{nolocal}
    \end{lemma}
\end{frame}

%----------------------------------------------------------------

\begin{frame}{Proof of lemma 1}
        \begin{itemize} 
            \item<1-> First clause ``$[U] \subseteq A \cup A_{l, s(n)}$'' follows from $1$ and $2$.
            \item<2-> $|s| = n+1 \leq m$, $n=0$: \begin{itemize}
                \item<3-> Suppose $P^{\emptyset}(U)$ holds, there is $\tau$ winning for II\@. 
                \item<4-> $2$ and $3$ implies $\exists d \in D \cap \tau \ \forall x \supset d \ x \in [\tau] \rightarrow 
                x \in [R^d]$. 
                \item<5-> So $\tau_d$ is winning for II in $G(U_d \cap R^d, A)$, contradiction.
            \end{itemize} 
            \item<6-> By induction hypothesis we have to build for $n>1$, $\Hat{U}$, $\Hat{D}$ and 
            $\{\Hat{R}^d : d \in \Hat{D}\}$ locally witnessing $P^{s[n-1]}(\hat{S})$.
            \item<7-> We want to define $\hat{U}$ such that $\hat{U}$ falls out of the $R^d$ and 
            hence ends up in $A$ (we suppose $m-n$ odd). 
            \item<8-> We use $\Sigma^1_{|s|}$-$\mathsf{AC}_0$ to pick up witnesses for $P^{s[n-1]}(\hat{R}^d)$
            which will be our $\hat{U}_d \cap \hat{R}^d$.

        \end{itemize}
\end{frame}

%----------------------------------------------------------

\begin{frame}

    \begin{definition}
        We say that $P^s(S)$ fails everywhere if $P^s(S_p)$ fails for every $p \in S$. This is a $\Pi^1_{|s|+2}$ 
        sentence.
    \end{definition}
    \begin{lemma}[2]<2->
        If $P^s(S)$ fails, then there is a quasistrategy $W$ for I if $l$ is odd in $S$ such that 
        $P^s(W)$ fails everywhere.\label{failure}
    \end{lemma}
\end{frame}

%----------------------------------------------------------------

\begin{frame}{Proof of lemma 2}
    \begin{itemize}
        \item<1-> $|s| = 0$, then II does not have winning strategy and we take I's non-loosing 
        quasistrategy as $W$ (Using $\Pi^1_2$-$\mathsf{CA}_0$).
        \item<2-> $|s| = n + 1$ (and $l$ even), we use  $\Pi^1_{|s|+2}$-$\mathsf{CA}_0$ to set
        \begin{align*}
            d \in D \leftrightarrow d \in S \land P^s(S_d) \land \lnot P^s(S_{d[|d|-1]}). \quad 
            \text{($d$ is minimal)}
        \end{align*}
        \item<3-> We will use $\Sigma^1_{|s|}$-$\mathsf{AC}_0$ to chose a witness $U^d$.
        \item<4-> We now consider the game $G(S,B)$ where 
        \begin{align*}
            B = \{x \in [S] \mid \exists d \in D \ d \subseteq x\}.
        \end{align*}
        \item<5-> Using preceding lemma, $G(S,B)$ is not a win for I.
        \item<6-> Again we define $W$ as non-losing II's quasistrategy (and use preceding lemma to 
        show it is as required).
    \end{itemize}
\end{frame}

%----------------------------------------------------------

\begin{frame}
    \begin{definition}
        For $|s| = n+1$, $W$ strongly witnesses $P^s(S)$ if, for all $p \in W$, $W_p$ witnesses $P^s(S_p)$, that 
        is, $W$ witnesses $P^s(S)$ and $P^{s[n]}(W)$ fails everywhere. This is a $\Pi^1_{|s+1|}$ 
        sentence.
    \end{definition}
\end{frame}

%----------------------------------------------------------------

\begin{frame} 
    \begin{lemma}[3]
        If $P^s(S)$, then there is a $W$ that strongly witnesses it.\label{enpowerment}
    \end{lemma}
\end{frame}

%----------------------------------------------------------------

\begin{frame}{Proof of lemma 3}
    Take $U$ such that
    \begin{align*}
        [U] \subseteq A \cup A_{l, s(n)} \qquad \text{and} \qquad 
        P^{s[n]}(U) \text{ fails}.
    \end{align*}
    \pause
    Apply preceding lemma to get a $W$ such $P^{s[n]}(W)$ fails everywhere.
\end{frame}

%----------------------------------------------------------

\begin{frame}
    \begin{lemma}[4]
        If $|s| = n+1$, then at least one of $P^s(S)$ and $P^{s[n]}(S)$ holds.\label{binary}
    \end{lemma}
\end{frame}

%----------------------------------------------------------------

\begin{frame}{Proof of lemma 4}
    \begin{itemize}
        \item<1-> Reverse induction on $n < m$, $m-n$ odd, suppose $P^s(S)$ fails. 
        \item<2-> Using strong $\Sigma^1_{m+2}$-$\mathsf{DC}_0$, we define by induction a quasistrategy 
        $U$ for II in $S$ along with $D \subseteq S$ and $R^d$ for $d \in D$ showing that \begin{align*}
            U \text{ (locally) witnesses } P^{s[n]}(S) \text{ if } (n > 0) \ n=0.
        \end{align*}
        \item<3-> The method to ensure $x \in [U] \setminus_{d \in D} R^d$ implies $x \in \Bar{A}$ is as 
        follow: \begin{enumerate}
            \item<4-> We define $U$ such that $x \in \Bar{A} \cup A_{m-n-2, j}$, $\forall j < \omega$ such that 
            $x \not\in \Bar{A} \cup A_{m-n-2, }$.
            \item<5-> We make sure $x \not \in A_{m-n-1,s(n),j}$, $\forall j < \omega$ such that
            $x \not \in A_{m-n-1}$.
            \item<6-> We then use $\Bar{A} \cup A_{m-n-2} \setminus A_{m-n-1} \subseteq \Bar{A}$.
        \end{enumerate}
    \end{itemize} 
\end{frame}

%----------------------------------------------------------

\begin{frame}{Proof of the theorem}
    \begin{itemize}
        \item<1-> Suppose $m$ is odd and $G(A,T)$ is not a win for II\@; $P^{\emptyset}(T)$ fails.
        \item<2-> Take $W^{\emptyset}$ such that $P^{\emptyset}(W^{\emptyset})$ fails everywhere.
        \item<3-> We define a quasistrategy $U$ for I in $W^{\emptyset}$ by induction on $|p|$ for $p \in U$.
        In the same time, we use strong $\Sigma^1_3$-$\mathsf{DC}_0$ to define for $|p| = j+1$ a quasistrategy $W^p$ 
        for I such that \begin{align*}
            W^p \text{ strongly witnesses } P^{\langle j \rangle}(W^{p[j] }_p).
        \end{align*}
    \end{itemize}
\end{frame}

%----------------------------------------------------------

\begin{frame}{Proof of the theorem}
    \begin{itemize}
        \item<1-> Suppose then $p \in U$, $|p| = j+1$ and 
        $W^{p}$ has been defined. The child $q$ of $p$ in $U$ are those of $p$ in $W^p$.
        \item<2-> $P^{\emptyset}(W^p)$ fails everywhere and so, 
        \begin{align*} 
            P^{\emptyset}(W^p_q) \text{  fails for each child $q$ of $p$ in $U$.} 
        \end{align*}
        \item<3-> By preceding lemma $P^{\langle j \rangle}(W^p_q)$ and we choose a $W^q$ 
        that strongly witnesses it.
    \end{itemize}
\end{frame}
%----------------------------------------------------------

\begin{frame}
    Consider any play $x \in [U]$.
    \begin{itemize}
        \item<2->  For all $j$, \begin{align*}
            \forall j \ x \in [W^{x[j+1]}], \text{ which witnesses } P^{\langle j \rangle}.
        \end{align*}
        \item<3-> Then $\forall j \ x \in A \cup A_{m-1, j}$,
        \item<4-> But $\bigcap_{j < \omega} A_{m-1, j} = A_{m-1} \subseteq A$.
    \end{itemize}    
\end{frame}
    

%----------------------------------------------------------

\section*{}
\begin{frame}
    \huge{Thank you for your attention!}
\end{frame}
\end{document}
